% Resumo
\begin{resumo}
% Diminuir espaçamento entre título e texto
\vspace{-1cm}

% Texto do resumo: sem paragrafo, justificado, com espaçamento 1,5 cm
\onehalfspacing
\noindent 
%\begin{shaded*}
Dispositivos de \ac{IHC} estão cada vez mais presentes em nosso cotidiano, com telas de toque ou controles por gestos, permitindo uma utilização mais natural e intuitiva de \textit{Hardware} e \textit{Software}. Luvas Eletrônicas são um tipo de dispositivo \ac{IHC} que permite o controle de braços robóticos ou objetos em ambientes virtuais através de gestos da mão do usuário, porém são muito caras e isso torna seu uso inviável para aplicações não profissionais. Este trabalho apresenta a continuação do desenvolvimento de um protótipo de luva eletrônica elaborado por \citeonline{roversi}, utilizando-se a plataforma \textit{Arduino}, sensores de flexão e a \ac{IMU} \textit{MPU-9250}. As melhorias propostas são incluir a capacidade de detecção de movimentos de adução e abdução dos dedos e desvios radial e ulnar do pulso, além de reduzir o ruído proveniente da \ac{IMU} e dos sensores de flexão. Os movimentos do usuário serão mostrados na tela do computador através de um modelo \ac{3D} de uma mão humana utilizando a plataforma \textit{Unity} de desenvolvimento, de modo a verificar a correção dos dados capturados e enviados. A solução para a adição dos movimentos dos dedos foi limitada, não permitindo a movimentação do dedo médio. A solução para detecção dos movimentos do pulso foi satisfatória e exibe boa precisão. Os filtros aplicados nos sensores foram satisfatórios provendo boa redução de ruídos sem comprometer a responsividade dos movimentos.

%\end{shaded*}
% Espaçamento para as palavras-chave
\vspace*{.75cm}

% Palavras-chave: sem parágrafo, alinhado à esquerda
\noindent Palavras-chave: interação humano computador. luva eletrônica. arduino. unity. acelerômetro. giroscópio. magnetômetro. IMU. sensor flexão.\\
% Segunda linha de palavras-chave, com espaçamento.
%\indent\hspace{2cm}Palavra.

\end{resumo}