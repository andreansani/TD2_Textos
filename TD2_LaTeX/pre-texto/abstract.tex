% Abstract
\begin{abstract}
% Diminuir espaçamento entre título e texto
\vspace{-1cm}
% Texto do resumo, em inglês: sem paragrafo, justificado, com espaçamento 1,5 cm
\onehalfspacing
\noindent
Human-Computer Interaction devices are increasingly present in our daily lives, with touchscreens or gesture controls, allowing more natural and intuitive use of \textit {Hardware} and \textit {Software}. Data Gloves are a type of Human-Computer Interaction device that allows control of robotic arms or objects in virtual environments through hand gestures, but they are expensive and this makes their use unfeasible for non-professional applications. This work presents the further development of an data glove prototype made by \citeonline{roversi}, using the \textit{Arduino} platform, felxibility sensors and the \ac{IMU} \textit{MPU-9250}. We propose to include the ability to detect adduction and abduction movements of the fingers and radial and ulnar deviations of the wrist, in addition to reducing the noise coming from the \ac{IMU} and the flex sensors. User movements will be shown on the computer screen through a \ac{3D} model of a human hand using the \textit{Unity} development platform. The solution for the addition of finger movements was limited, not allowing movement of the middle finger. The solution for detecting wrist movements was satisfactory and shows good accuracy. The filters applied to the sensors were satisfactory providing good noise reduction without compromising the responsiveness of the movements.

% Espaçamento para as palavras-chave
\vspace*{.75cm}

% Palavras-chave: sem parágrafo, alinhado à esquerda
\noindent Keywords: human computer interaction. data glove. arduino. unity. accelerometer. gyroscope. magnetometer. IMU. flex sensor.\\
% Segunda linha de palavras-chave, com espaçamento.
%\indent\hspace{1.4cm} Keyword.

\end{abstract}